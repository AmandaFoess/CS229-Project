\section{Introduction}

The finished home price to land value ratio is a fundamental metric in land development,
representing how much of a house's total value comes from the land itself. This ratio
plays a pivotal role in evaluating the financial viability of real estate projects,
particularly in mass production housing. By analyzing and tracking this ratio across
various submarkets, developers can make more precise financial decisions
regarding raw land acquisition and project planning, aligning development strategies
with market conditions.

\[
	\text{Ratio}= \frac{ \text{Market retail sale price of a new finished home} }{
	\text{Vacant developed lot price the builder pays the developer} }
\]

This ratio is determined by two main components: (1) the market value of the finished
property, influenced by factors such as location, size, and sales of comparable
properties, and (2) the price of a vacant developed lot (VDL) that builders are willing
to pay, which is calculated using the residual land valuation method. A well-balanced
ratio signals that development costs align with the land’s worth, suggesting a
potentially profitable project. On the other hand, a skewed ratio could mean the
project is burdened with high costs or that the land’s value is underestimated, prompting
the need for further financial evaluation. For developers, understanding this ratio
across the broader market, submarkets, and individual builder preferences allows
for accurate lot pricing, ensuring builders can achieve their desired internal
profit margins.

Estimating this ratio accurately is no small task, as it requires navigating a range
of challenges. Real estate markets are inherently dynamic, influenced by fluctuating
economic conditions, interest rates, and local developments. Market-specific
factors, such as neighborhood appeal, proximity to amenities, and future development
potential, also significantly impact the ratio but are difficult to quantify
precisely. In addition, zoning regulations, land-use policies, and environmental
rules introduce further complexity. Traditional valuation methods, while
valuable, often rely on detailed financial forecasts and assumptions that can introduce
uncertainty.

This study applies a machine learning approach, using XGBoost, to predict the finished
home price to land value ratio at the early stages of a project. This proactive
approach shifts the focus from traditional reactive calculations—performed after
significant project data has been finalized—to early-stage predictions that
empower stakeholders to make informed decisions right from the outset.

The model integrates diverse data sources, including numeric inputs such as acreage,
estimated Vacant Developed Lot (VDL) sale prices, and geospatial details (latitude,
longitude, and sale date), as well as categorical features like land usage and
grantee mappings. By leveraging these inputs, the model uncovers complex
patterns and relationships that influence the predicted finished lot price to
land value ratio.

This research contributes to the development of more efficient land valuation
practices, aiding in the supply of affordable housing and addressing housing shortages
in high-demand areas. By improving the accuracy of this critical ratio and
adapting it to localized submarket conditions, the study offers a powerful tool for
strategic decision-making in land development.